% Author: Dominik Harmim <harmim6@gmail.com>

\documentclass[a4paper, 10pt, twocolumn]{article}

\usepackage[british]{babel}
\usepackage[utf8]{inputenc}
\usepackage[T1]{fontenc}
\usepackage[left=2cm, top=2cm, text={17cm, 25cm}]{geometry}
\usepackage[unicode, colorlinks, hypertexnames=false, citecolor=red]{hyperref}
\usepackage{times}
\usepackage{graphicx}
\usepackage{amsmath}

\setlength{\parindent}{0pt}
\setlength{\parskip}{.5 \bigskipamount}


\begin{document}
    \twocolumn[
        \begin{@twocolumnfalse}
            \begin{center}
                {\Large
                    Brno University of Technology \\
                    Faculty of Information Technology \\
                }
                {\includegraphics[width=.4 \linewidth]{img/FIT_logo.pdf}} \\

                {\LARGE
                    Cryptography \\
                    2.~Project\,--\,RSA \\[.4cm]
                }

                {\large
                    Dominik Harmim (xharmi00) \\
                    \texttt{xharmi00@stud.fit.vutbr.cz} \\
                    \today
                }
            \end{center}
        \end{@twocolumnfalse}
    ]


    \section{Introduction}

    The goal of this project is to implement the RSA algorithm\,---\,an
    asymmetry cryptography algorithm. The program that should be created
    should be able to generate parameters of the RSA, encrypt and decrypt
    messages, and break the algorithm using factorisation of the public
    modulus. The program is implemented in~C++ and it is used arithmetic
    library GMP\footnote{Arithmetic library
    \textbf{GMP}\,--\,\url{https://gmplib.org}.} for the computation with
    large numbers. In the following chapters, there are briefly described
    the methods and algorithms used. Appropriate bibliography and other
    sources are mentioned as well.


    \section{Encryption and Decryption}

    An encryption and decryption is implemented by~\cite{pkcrypt}. The
    private transformation~$ D $~and the public transformation~$ E $~are
    defined as follows: $ D(c) = m = c^d\,\mod n $; $ E(m) = c = m^e\,\mod
    n $, where~$ m $~is a~decrypted message, $ c $~is an encrypted message,
    $ e $~is the public exponent, $ d $~is the private exponent, and
    $ n $~is the public modulus.


    \section{RSA Parameters Generation}

    The generation of RSA parameters is implemented according
    to~\cite{pkcrypt}. At first, two random prime numbers ($ p $~and~$ q $)
    are generated and $ n = p \cdot q $ and $ h = (p - 1) \cdot (q - 1) $ is
    computed. Then, it is chosen~$ e $~to be an integer in range
    $ (2, h - 1) $ with $ GCD(e, h) = 1 $. Further, it is found the
    multiplicative inverse~$ d $~of~$ e $, modulo~$ h $. Now, $ n $~and~$ e
    $~are public and~$ d $, $ p $, $ q $, $ h $ are secret.

    The computation of a~greatest common divisor (the $ GCD $ function)
    is done using the Euclid's algorithm. The computation of a~multiplicative
    inverse is done using the extended Euclid's algorithm. These algorithms
    are in detail described in~\cite{pkcrypt}.

    For the generation of the prime numbers, it is used the probabilistic
    approach using Solovay-Strassen test with the computation of the
    Jacobi symbol. These algorithms are explained and demonstrated
    in~\cite{pkcrypt, solovay, jacobi}.


    \section{Breaking the Algorithm}

    Breaking of the algorithm is achieved using the factorisation of the
    public modulus. Once the factorisation is done, one of the secret
    prime numbers is obtained and the other one is then easily calculated
    as well as the private exponent. So, the private key is obtained.

    Factorisation of the public modulus is implemented using the
    Pollard Rho Brent Integer Factorisation, see~\cite{brent, brentOnline}.


    \section{Conclusion}

    Within this project, the RSA algorithm has been successfully
    implemented in a~desirable way. It was successfully tested on
    computer \texttt{merlin.fit.vutbr.cz}. It was experimentally
    verified that the program can correctly generate keys for the public
    modulus up to~4096 bits length and that it can break the algorithm
    for the public modulus up to~100 bits length quite fast. The program
    works properly also for longer keys but, as expected, it will take
    more time.


    \bibliographystyle{englishiso}
    \renewcommand{\refname}{Bibliography}
    \bibliography{doc}
\end{document}
